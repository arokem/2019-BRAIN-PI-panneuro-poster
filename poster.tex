%%%%%%%%%%%%%%%%%%%%%%%%%%%%%%%%%%%%%%%%%
% a0poster Landscape Poster
% LaTeX Template
% Version 1.0 (22/06/13)
%
% The a0poster class was created by:
% Gerlinde Kettl and Matthias Weiser (tex@kettl.de)
%
% This template has been downloaded from:
% http://www.LaTeXTemplates.com
%
% License:
% CC BY-NC-SA 3.0 (http://creativecommons.org/licenses/by-nc-sa/3.0/)
%
%%%%%%%%%%%%%%%%%%%%%%%%%%%%%%%%%%%%%%%%%

%-------------------------------------------------------------------------------
%	PACKAGES AND OTHER DOCUMENT CONFIGURATIONS
%-------------------------------------------------------------------------------

\documentclass[a0, landscape]{a0poster}

\usepackage{multicol} % This is so we can have multiple columns of text side-by-side
\columnsep=100pt % This is the amount of white space between the columns in the poster
\columnseprule=3pt % This is the thickness of the black line between the columns in the poster

\usepackage[svgnames]{xcolor} % Specify colors by their 'svgnames', for a full list of all colors available see here: http://www.latextemplates.com/svgnames-colors

%\usepackage{times} % Use the times font
\usepackage{palatino} % Uncomment to use the Palatino font
\usepackage{graphicx} % Required for including images
\graphicspath{{figures/}} % Location of the graphics files
\usepackage{booktabs} % Top and bottom rules for table
\usepackage[font=small,labelfont=bf]{caption} % Required for specifying captions to tables and figures
\usepackage{amsfonts, amsmath, amsthm, amssymb} % For math fonts, symbols and environments
\usepackage{wrapfig} % Allows wrapping text around tables and figures

\usepackage{listings}
\usepackage{color}

\definecolor{dkgreen}{rgb}{0,0.6,0}
\definecolor{gray}{rgb}{0.5,0.5,0.5}
\definecolor{mauve}{rgb}{0.58,0,0.82}

\lstset{
  %frame=tb,
  language=Python,
  aboveskip=2mm,
  belowskip=0mm,
  showstringspaces=false,
  columns=flexible,
  basicstyle={\small\ttfamily},
  numbers=none,
  numberstyle=\tiny\color{gray},
  keywordstyle=\color{blue},
  commentstyle=\color{dkgreen},
  stringstyle=\color{mauve},
  breaklines=true,
  breakatwhitespace=true,
  tabsize=3,
  framexleftmargin=15pt
  }


\begin{document}

%----------------------------------------------------------------------------------------
%	POSTER HEADER
%----------------------------------------------------------------------------------------

% The header is divided into three boxes:
% The first is 55% wide and houses the title, subtitle, names and university/organization
% The second is 25% wide and houses contact information
% The third is 19% wide and houses a logo for your university/organization or a photo of you
% The widths of these boxes can be easily edited to accommodate your content as you see fit

\begin{minipage}[b]{0.85\linewidth}
\veryHuge \color{NavyBlue} \textbf{PanNeuro: leveraging a community-based approach for big data neuroscience } \color{Black}\\ % Title
%\Huge\textit{An Exploration of Complexity}\\[1cm] % Subtitle
\huge \textbf{Ariel Rokem\textsuperscript{1,2}, Joe Hamman \textsuperscript{3}, Ryan Abernathy \textsuperscript{4} \&  \textsuperscript{5}}\\ % Author(s)
\Large 1. The eScience Institute, 2. Computational Neuroscience Center 4. Dept. of Physiology and Biophsyics, Univ. of Washington \\
3. NCAR, 4. Columbia University % University/organization
\Large Contact: \texttt{arokem@uw.edu} $|$ Download: \texttt{http://arokem.org/presentations/brainpi-panneuro-2019}
\end{minipage}

\begin{minipage}[b]{0.19\linewidth}
\includegraphics[width=10cm]{UWlogo.png}
\end{minipage}

\vspace{0.5cm} % A bit of extra whitespace between the header and poster content

%----------------------------------------------------------------------------------------

\begin{multicols}{3} % This is how many columns your poster will be broken into, a poster with many figures may benefit from less columns whereas a text-heavy poster benefits from more

%----------------------------------------------------------------------------%	Introduction
%----------------------------------------------------------------------------

\section*{Introduction}



\vfill
\columnbreak

%----------------------------------------------------------------------------%
%----------------------------------------------------------------------------
\color{Navy}

\section*{Cloud computing}

\vfill
\columnbreak

\section*{Results}

%----------------------------------------------------------------------------
%	MATERIALS AND METHODS
%----------------------------------------------------------------------------

\color{SaddleBrown} % SaddleBrown color for the conclusions to make them stand out
\section*{Conclusions}
\large
\begin{itemize}

\item Conclusion, the first

\item Conclusion, the second

\item Conclusion, the third

\end{itemize}

\color{DarkSlateGray} % Set the color back to DarkSlateGray for the rest of the content

%---------------------------------------------------------------------------	REFERENCES
%----------------------------------------------------------------------------

\nocite{*} % Print all references regardless of whether they were cited in the poster or not
\bibliographystyle{plain} % Plain referencing style
\footnotesize \bibliography{poster} % Use the example bibliography file sample.bib

%----------------------------------------------------------------------------%	ACKNOWLEDGEMENTS
%----------------------------------------------------------------------------
\subsection*{Acknowledgements} \footnotesize


\includegraphics[height=2.6cm]{NIBIB.png}\\
\includegraphics[height=2.6cm]{SloanLogo.png}
\includegraphics[height=2.6cm]{MooreFdn.png}
\includegraphics[height=2.6cm]{eSciencelogo.png}
%----------------------------------------------------------------------------

\end{multicols}
\end{document}
